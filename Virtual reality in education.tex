% !TeX spellcheck = de_DE
% Metódy inžinierskej práce

\documentclass[10pt,twoside,slovak,a4paper]{article}

\usepackage[slovak]{babel}
%\usepackage[T1]{fontenc}
\usepackage[IL2]{fontenc} % lepšia sadzba písmena Ľ než v T1
\usepackage[utf8]{inputenc}
\usepackage{graphicx}
\usepackage{url} % príkaz \url na formátovanie URL
\usepackage{hyperref} % odkazy v texte budú aktívne (pri niektorých triedach dokumentov spôsobuje posun textu)

\usepackage{cite}
%\usepackage{times}

\pagestyle{headings}

\title{Virtuálna realita vo vzdelavacom systéme\thanks{Semestrálny projekt v predmete Metódy inžinierskej práce, ak. rok 2020/21, vedenie: Martin Sabo}} % meno a priezvisko vyučujúceho na cvičeniach

\author{Patrik Kozlík\\[2pt]
	{\small Slovenská technická univerzita v Bratislave}\\
	{\small Fakulta informatiky a informačných technológií}\\
	{\small \texttt{xkozlik@stuba.sk}}
	}

\date{\small 2. november 2020} % upravte



\begin{document}

\maketitle

\begin{abstract}
Dnešná doba je primárne ovplyvnená vývojom technológii a novými výskummy z oblasti informatiky. Informatika stále káždým rokom zasahuje
viac do rôznych iných vedných oblastí. Počas tohoto rozvoja vnikla aj technológia virtuálnej reality, ktorá sa preslávila najmä vďaka jej
implemetácie do počítačových hier. Pri jej vývoji sa časom došlo na to, že by sa virtuálna realita dala zakomponovať aj do vzdelávania. 
Pri čom sa budem primárne zameriavať na jej implementáciu nie len do škôl, škôlok, zariadení pre ľudí so špeciálnymi potrebamy ale aj do múzeí. V závere uvediem vlastný návrh prepojenia virtualnej reality, múzea a študentov umeleckých vysokých škôl.   	
\end{abstract}



\section{Úvod}

Virtuálna realita v nedávnej dobe dosiahla veľkého rozmachu. Stala sa viac dostupná pre väčšie počty ľudí a hlavne čo je najdôležitejšie 
častejšie a častejšie sa s ňou začali stretávať aj obyčajný ľudia mimo IT sektoru. Vďaka tomu sa stala komerčná a dosiahla väčšej pozornosti,
čo znamená že dostala viac možností a financií na rozvoj. 

Avšak ohľadom virtuálnej reality v dnešnej dobe sa nám naskytáva otázka. Vieme že je využitá vo veľkom počte vďaka komercii ale je aj dostatočne využívaná vo vzdelavacom systéme na takej úrovni ako by mala byť?~\ref{realita}

Základom je zistenie výhod a nevýhod virtuálnej reality aby sme vedeli s čím pracujeme a čo môže narušiť alebo urýchliť proces 
jej implementácie do vzdelávania.~\ref{procon}

Jednou z hlavných častí je už samotná implementácia virtuálnej reality do škôl, či do zariadení pre ľudí so špeciálnymi potrebamy,
kde môže byť veľmi prospešná a náučná.~\ref{školy}

No aby sme nezabudli vzdelávací proces neprebieha len v školách ale aj v múzeách, ktoré sú založené na približovaní histórie návštevníkovi, čo nám vie virtuálna realita predstaviť vizuálnou a zábavnejšou formou.~\ref{muzea}

Potenciálnym prepojením virtuálnej reality v múzeách a vzdelávania v školách by sme mohli dosiahnuť veľmi žiaduce účinky na vzdelanosť študentov a zvýšiť ich motiváciu k štúdiu s pomocou tejto zábavnej a náučnej formy.~\ref{produkt}

\section{Virtualna realita} \label{realita}

 

\begin{figure*}[tbh]

\end{figure*}



\section{Výhody a nevýhody virtuálnej reality} \label{procon}


\subsection{Výhody} \label{procon:vyhody}



\subsection{Nevyhody} \label{procon:nevyhody}


\section{Školy a pomocné zariadenia} \label{školy}
Už od začiatkov by s sme mali deti vzdelávať čo najlepšou a najprínosnejšou formou, jednou z nich je virtuálna realita. S jej pomocou môžeme dosiahnuť zábavnejší a lahšie memorizovatelnejší druh vzdelávania. No prospešná nieje len pre školy s normálnym zameraním ale je prospešná aj pre zariadenia so špeciálnym zameraním. 

\subsection{Školy} \label{školy:školy}
V základných školách sa virtuálna realita ešte len začína pomaly využívať a nieje bežným štandardom pre každú školu. Jedným z dôvodou sú financie, virtuálna realita nieje práve najlacnejšia technológia a nieje jednoduché zaobstarať triedu vybavenú touto technológiou. Avšak, myslím si že časom bude štandardom. Už teraz môžeme vidieť projekty ako je magic Mobius strip~\cite{Math} alebo 3D-VRLE~\cite{Physics}, ktoré vykazujú výsledky priamo zo škôl. 

Projekt magic Mobius strip je veľmi užitočný a prospešný návrh výučby matematiky, primárne geometrie, na zakladných školách. Projekt bol už v roku 2016 oskúšaný na pár zakladných školách v Číne, pri čom obdržal len pozitívne ohlasy. Klúčovou zložkou projektu je príslušenstvo, ktoré pozostáva z tabule, iPad ovládacieho systému, audia, VR okuliarov a vysielača VR siete. Pokus bol spravený na 32 študentoch vo veku od 10 do 12 rokov. Bolo viditeľné, že pri štúdiu geometrie niektrorí študenti nemali záujem, no v momente ako sa zapojil do výučby magic Mobius strip, všetko sa náhle zmenilo. Učili sa interaktívnou formou, mohli vytvárať vlastný útvar ,takzvaný mobiov list, a dokonca si aj zahrať interaktívnu hru.~\cite{Math}

Ďalšim úžasnym projektom je 3D-VRLE. Projekt zameraný na fyziku, avšak pre zmenu sa tentoraz projekt zameriava už aj na zvýšenie motivácie pre daný predmet. Štúdiu uskutočnili spomedzi 65 študentiek, ktoré rozdelili na dve skupiny. Jedna skupina mala k dispozícii pri štúdiu bežné veci knihy, tabulu či prezentácie, no druhá skupina mala 3D-VRLE. Tento enviroment je tvorený balíkom digitálnych simulácii, animácii či interaktívnych demonštrácii fyziky vo virtuaĺnych laboraóriách. Pri študentoch, ktorý využívali 3D-VRLE bolo jasne preukázané zvýšenie nie len vedomostí, ktoré nadobudli a lepšie si zapamätali, ale aj zvýšenie záujmu o fyziku.~\cite{Physics} 

Podobných projektov, ktoré spájajú virtuálnu realitu a školy je mnoho. Toto sú len drobné príklady, no už len podľa týchto príkladov vieme povedať, že jej využívanie v školách má veľa benefitov od lepšieho zapamätávania učiva až po zvýšenie motivácie. Motivácia je nosným kameňom celého vzdelávanie. Ak by sme vedeli dobre implementovať virtuálnu realitu už v skorých rokoch štúdia na školách do bežnej výučby, mali by sme viac motivovaných študentov, ktorý by sa chceli vzdelávať v oblastiach, ktoré im táto virtuálna realita priblíži.  


\subsection{Zariadení pre ľudí so špeciálnymi potrebamy} \label{školy:zariadenia}
Virtuálna realita nemusí byť prospešná len v školách pri učení, no jej ďalšie úžasne využitie je aj na miestach kde by ste to naozaj neočakávali. Sú to zariadenia pre ľudí so špecialnymi potrebami. Jedným z príkladov sú osoby trpiace autizmom, ľudia čo trpia touto poruchou majú problémy s komunukáciou a socializáciou s okolím. No ak sa táto porucha indentifikuje a daný človek s ňou chce niečo spraviť sú možnosti ako. Práve jednou z možností je projekt half-CAVE~\cite{Autism}, ktorý využíva virtuálnu realitu na vzdelávanie týchto ľudí.

Tento úžasný projekt s názvom half-CAVE nie je založený na rovnakých princípoch ako ostatné projekty. Nesnaží sa až v takej miere motivovať ako ostatné, no skôr pomáhať ľudom s autizmom pri ich adaptácii do bežného života. Projekt je tvorený 6 simuláciamy, ktoré človeka naučia ako zvládať emócie, stres v bežných situáciách ako je ranné vstávanie. Taktiež učia ako využívať relaxačné stratégie v napätých situáciach. No tento tréning nie je jednoduché zrealizovať, kvôli zložtosti virtuálneho prostredia. Prostredie potrebuje, jednu miestnosť, projektor, veľmi výkonný procesor a grafickú kartu, kameru, zrkadlo a pre priblíženie 128GB RAM.~\cite{Autism}

Ako vidíme virtualna realita je veľmi prospešná aj pri pomoci ľuďom so špecialnymi potrebami. Vďaka nej majú väčšiu možnosť na adaptáciu do normálneho života. Samozrejme človeka nevylieči úplne, ale už len tým že zlepší človeku kvalitu života, čo i len o trochu má veľký potenciál do budúcna. Momentálne tieto tréningy nie sú moc rozšírené, no ako budú časom ľudia zisťovať benefity tejto technológie budeme o nej frekventovanejšie počuť ako doteraz.   

\section{Virtuálna realita v múzeách} \label{muzea}


\section{Virtuálna realita v múzeách v spojení so sochárstvom} \label{produkt}


\section{Záver} \label{zaver} % prípadne iný variant názvu



%\acknowledgement{Ak niekomu chcete poďakovať\ldots}


% týmto sa generuje zoznam literatúry z obsahu súboru literatura.bib podľa toho, na čo sa v článku odkazujete
\bibliography{literatura}
\bibliographystyle{plain} % prípadne alpha, abbrv alebo hociktorý iný
\end{document}
