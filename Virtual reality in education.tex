% !TeX spellcheck = de_DE
% Metódy inžinierskej práce

\documentclass[10pt,twoside,slovak,a4paper]{article}

\usepackage[slovak]{babel}
%\usepackage[T1]{fontenc}
\usepackage[IL2]{fontenc} % lepšia sadzba písmena Ľ než v T1
\usepackage[utf8]{inputenc}
\usepackage{graphicx}
\usepackage{url} % príkaz \url na formátovanie URL
\usepackage{hyperref} % odkazy v texte budú aktívne (pri niektorých triedach dokumentov spôsobuje posun textu)

\usepackage{cite}
%\usepackage{times}

\pagestyle{headings}

\title{Virtuálna realita vo vzdelavacom systéme\thanks{Semestrálny projekt v predmete Metódy inžinierskej práce, ak. rok 2020/21, vedenie: Martin Sabo}} % meno a priezvisko vyučujúceho na cvičeniach

\author{Patrik Kozlík\\[2pt]
	{\small Slovenská technická univerzita v Bratislave}\\
	{\small Fakulta informatiky a informačných technológií}\\
	{\small \texttt{xkozlik@stuba.sk}}
	}

\date{\small 2. november 2020} % upravte



\begin{document}

\maketitle

\begin{abstract}
Dnešná doba je primárne ovplyvnená vývojom technológii a novými výskummy z oblasti informatiky. Informatika stále káždým rokom zasahuje
viac do rôznych iných vedných oblastí. Počas tohoto rozvoja vnikla aj technológia virtuálnej reality, ktorá sa preslávila najmä vďaka jej
implemetácie do počítačových hier. Pri jej vývoji sa časom došlo na to, že by sa virtuálna realita dala zakomponovať aj do vzdelávania. 
Pri čom sa budem primárne zameriavať na jej implementáciu nie len do škôl, škôlok, zariadení pre ľudí so špeciálnymi potrebamy ale aj do múzeí. V závere uvediem vlastný návrh prepojenia virtualnej reality, múzea a študentov umeleckých vysokých škôl.   	
\end{abstract}



\section{Úvod}

Virtuálna realita v nedávnej dobe dosiahla veľkého rozmachu. Stala sa viac dostupná pre väčšie počty ľudí a hlavne čo je najdôležitejšie 
častejšie a častejšie sa s ňou začali stretávať aj obyčajný ľudia mimo IT sektoru. Vďaka tomu sa stala komerčná a dosiahla väčšej pozornosti,
čo znamená že dostala viac možností a financií na rozvoj. 

Avšak ohľadom virtuálnej reality v dnešnej dobe sa nám naskytáva otázka. Vieme že je využitá vo veľkom počte vďaka komercii ale je aj dostatočne využívaná vo vzdelavacom systéme na takej úrovni ako by mala byť?~\ref{realita}

Základom je zistenie výhod a nevýhod virtuálnej reality aby sme vedeli s čím pracujeme a čo môže narušiť alebo urýchliť proces 
jej implementácie do vzdelávania.~\ref{procon}

Jednou z hlavných častí je už samotná implementácia virtuálnej reality do škôlok, škôl, či do zariadení pre ľudí so špeciálnymi potrebamy,
kde môže byť veľmi prospešná a náčná.~\ref{školy}

No aby sme nezabudli vzdelávací proces neprebieha len v školách ale aj v múzeách, ktoré sú založené na približovaní histórie návštevníkovi, čo nám vie virtuálna realita predstaviť zábavnejšou a vizuálnou formou.~\ref{muzea}

Potenciálnym prepojením virtuálnej reality v múzeách a vzdelávania v školách by sme mohli dosiahnuť veľmi žiaduce účinky na vzdelanosť študentov a zvýšiť ich motiváciu k štúdiu s pomocou tejto zábavnej formy.~\ref{produkt}

\section{Virtualna realita} \label{realita}

 

\begin{figure*}[tbh]

\end{figure*}



\section{Výhody a nevýhody virtuálnej reality} \label{procon}


\subsection{Výhody} \label{procon:vyhody}



\subsection{Nevyhody} \label{procon:nevyhody}

\paragraph{Veľmi dôležitá poznámka.}
Niekedy je potrebné nadpisom označiť odsek. Text pokračuje hneď za nadpisom.



\section{Škôlky, školy a pomocné zariadenia} \label{školy}




\section{Virtuálna realita v múzeách} \label{muzea}


\section{Virtuálna realita v múzeách v spojení so sochárstvom} \label{produkt}


\section{Záver} \label{zaver} % prípadne iný variant názvu



%\acknowledgement{Ak niekomu chcete poďakovať\ldots}


% týmto sa generuje zoznam literatúry z obsahu súboru literatura.bib podľa toho, na čo sa v článku odkazujete
\bibliography{literatura}
\bibliographystyle{plain} % prípadne alpha, abbrv alebo hociktorý iný
\end{document}
